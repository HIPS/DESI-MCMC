%%%%%%%%%%%%%%%%%%%%%%%%%%%%%%%%%%%%%%%%%%%%%%%%%%%%%%%%%%%%%%%%%%
%%%%%%%% ICML 2015 EXAMPLE LATEX SUBMISSION FILE %%%%%%%%%%%%%%%%%
%%%%%%%%%%%%%%%%%%%%%%%%%%%%%%%%%%%%%%%%%%%%%%%%%%%%%%%%%%%%%%%%%%

% Use the following line _only_ if you're still using LaTeX 2.09.
%\documentstyle[icml2015,epsf,natbib]{article}
% If you rely on Latex2e packages, like most moden people use this:
\documentclass{article}

% use Times
\usepackage{times}
% For figures
\usepackage{graphicx} % more modern
%\usepackage{epsfig} % less modern
\usepackage{subfigure} 
\usepackage{amsmath,amsfonts,amssymb,bbm}

% For citations
\usepackage{natbib}

% For algorithms
\usepackage{algorithm}
\usepackage{algorithmic}

% As of 2011, we use the hyperref package to produce hyperlinks in the
% resulting PDF.  If this breaks your system, please commend out the
% following usepackage line and replace \usepackage{icml2015} with
% \usepackage[nohyperref]{icml2015} above.
\usepackage{hyperref}

% Packages hyperref and algorithmic misbehave sometimes.  We can fix
% this with the following command.
\newcommand{\theHalgorithm}{\arabic{algorithm}}
\DeclareMathOperator{\Tr}{Tr}
\newcommand{\R}{\mathbbm{R}}
\newcommand{\mba}{\mathbf{a}}
\newcommand{\mbb}{\mathbf{b}}
\newcommand{\mbx}{\mathbf{x}}
\newcommand{\mbxt}{\tilde{\mathbf{x}}}
\newcommand{\Sigmat}{\tilde{\Sigma}}
\newcommand{\mbz}{\mathbf{z}}
\newcommand{\mbw}{\mathbf{w}}
\newcommand{\mcN}{\mathcal{N}}
\newcommand{\mcP}{\mathcal{P}}
\newcommand{\eps}{\epsilon}
\newcommand{\trans}{\intercal}
\newcommand{\Ut}{\tilde{U}}
\DeclareMathOperator*{\argmax}{arg\,max}
\newcommand{\angstrom}{\textup{\AA}}
\newcommand{\red}[1]{\textcolor{red}{#1}}


% Employ the following version of the ``usepackage'' statement for
% submitting the draft version of the paper for review.  This will set
% the note in the first column to ``Under review.  Do not distribute.''
\usepackage{icml2015} 

% Employ this version of the ``usepackage'' statement after the paper has
% been accepted, when creating the final version.  This will set the
% note in the first column to ``Proceedings of the...''
%\usepackage[accepted]{icml2015}


% The \icmltitle you define below is probably too long as a header.
% Therefore, a short form for the running title is supplied here:
\icmltitlerunning{Model of Quasar Spectroscopy}

\begin{document} 

\twocolumn[
\icmltitle{A Stochastic Process Model of Quasar Spectroscopy} %# \\
%Inference of Red Shift from Quasar Photometry }

% It is OKAY to include author information, even for blind
% submissions: the style file will automatically remove it for you
% unless you've provided the [accepted] option to the icml2015
% package.
\icmlauthor{Andrew Miller}{acm@seas.harvard.edu}
\icmladdress{Harvard University,
            33 Oxford St, Cambridge, MA USA}
%\icmlauthor{Your CoAuthor's Name}{email@coauthordomain.edu}
%\icmladdress{Their Fantastic Institute,
%            27182 Exp St., Toronto, ON M6H 2T1 CANADA}

% You may provide any keywords that you 
% find helpful for describing your paper; these are used to populate 
% the "keywords" metadata in the PDF but will not be shown in the document
\icmlkeywords{boring formatting information, machine learning, ICML}

\vskip 0.3in
]

\begin{abstract} 
We present a method to measure the red-shift of quasars from photometric observations.  Our method treats the unknown spectrum of a quasar as a latent stochastic process and uses statistical inference to infer the unknown structure.  Our model leverages a small number of existing examples of full quasar spectra with known red-shift to build a structured stochastic process prior distribution over unknown spectra.  We then use Bayesian inference to infer red-shift from a typical 5-band photometric sample of astronomical imagery, the so called ``photo-z'' problem. 
\end{abstract} 

\section{Introduction}

Enormous amounts of astronomical data are collected to \red{cite sloan digital sky survey, other astronomical surveys}.  Among this collection are measurements of spectral energy distribution (SED) of a light source (e.g.~a star, galaxy, quasar).  
The spectrum of an object carries information about properties of the particular object, including effective temperature, type, red-shift, and chemical makeup.  

However, measurements of astronomical spectra are produced by instruments at widely varying resolutions.  
Spectroscopic measurements can resolve noisy measurements of the spectral energy distribution (SED) of an object (e.g.~a star, galaxy, or quasar) in finer detail than broadband photometric measurements.  For example, the Baryonic Oscillation Spectroscopic Survey \cite{dawson2013baryon} samples measurements at over four thousand wavelengths between 3,500 and 10,500 $\angstrom$.  In contrast, the photometry from the Sloan Digital Sky Survey (SDSS) \red{cite sloan}, gathers broadband photometric measurements in the u,g,r,i, and z bands.  These measurements are the weighted average response over a large swath of the spectrum \red{refer to plot}. 

Photometric measurements, however, are available for a larger number of sources, including objects that are fainter and possibly at extremely high red-shift (farther away).  This work focuses on extracting information from observations of light sources by jointly modeling spectroscopic and photometric data.  In particular, we we focus on measuring the red-shift of quasars for which we only have photometric observations.  Quasars, or quasi-stellar radio sources, are extremely distant and energetic sources of electromagnetic radiation that can exhibit high red-shift \red{cite something here}.  Identifying and measuring the red-shift of a quasars from photometric data is a necessary task due to the widespread availability of large photometric surveys.  Photometric estimates of red-shift have the potential to guide the study of certain quasars with higher resolution instruments.  Furthermore, accurate models can aid identification and classification of faintly observed quasars in such a large photometric survey.  Study of distant quasars allow astronomers to observe the universe as it was many billions of years ago \red{need lots of references..}

In this paper, we describe a probabilistic model that jointly describes both high resolution spectroscopic data and low resolution photometric observations of quasars in terms of their latent spectral energy distribution, luminosity, and red-shift.  We model a quasar's spectral energy as a hidden variable, and describe a fully Bayesian inference procedure to compute the marginal probability distribution of a quasar's red-shift given observed photometric fluxes and their uncertainties. 

\section{Background}
The spectral energy distribution of an object describes the distribution of energy it radiates as a function of wavelength.  
For example, most stars are well modeled as a blackbody, so their spectral radiance is well modeled by Planck's law, which describes a parametric form for the spectral energy distribution.  Resolvable stars, however, tend to be close enough that they are never observed at a measurable redshift. 
Quasars, on the other hand, have a complicated spectral energy distribution characterized by some salient features (\red{mention Ly-$\alpha$ and Lyman forest?}).  
Furthermore, quasars can be much more luminous and at a much higher red-shift.  
The red-shift effectively stretches an object's rest frame spectral energy distribution, skewing its mass toward higher wavelengths.  
Denoting the rest-frame SED of a quasar $n$ as a function, $f_n^{(rest)} : \mathcal{W} \rightarrow \R_+$, the affect of red-shift on our observations is summarized by the relationship 
\begin{align}
  f_n^{(obs)}(\lambda) &\propto f_n^{(rest)}(\lambda \cdot (1 + z_n)) \, .
\end{align}
An example of an observed quasar spectrum and it's de-redshifted rest frame spectrum can be found in \red{figure with an example}


\subsection{Related work}
The problem of estimating the red-shift of a source (galaxy or quasar) is known as ``photo-$z$''.  There have been many statistical and machine learning methods developed to tackle this problem. These roughly break down into template-based and regression-based methods \cite{walcher2011fitting}.  Regression-based (or empirical) methods for photo-$z$ fit a functional relationship between a set of photometric features and red-shift value.  

Probabilistic models for inferring red-shift from photometry has also had some success.  \cite{benitez2000bayesian} presents a thorough summary of Bayesian methods for photometric red-shift estimation from spectral templates.  

\cite{budavari2001photometric} and \cite{richards2001photometric} go into specific detail for SED model based photometric redshifts for quasars.  They compare a bunch of methods. Particularly, they describe an algorithm for reconstruction a quasar spectrum template from photometric observations and spectroscopic redshifts.  It seems sorta like a dynamic K-means/EM algorithm (they add spectral types as needed), and does a decent job reconstructing bumps where the emission lines are.  

\red{Others to mention: }
\begin{itemize}
\item \cite{bovy2012photometric} - extreme deconvolution method

\item \cite{suzuki2006quasar} (QUASAR SPECTRUM CLASSIFICATION WITH PRINCIPAL COMPONENT ANALYSIS (PCA): EMISSION LINES IN THE Ly$\alpha$ FOREST)

\item \cite{brescia2013photometric} use a multi-layer perceptron (four layers) regression setting on a combination of SDSS (from the DR7QSO dataset, I believe), UKIDSS, and WISE photometric datasets, comparing photo-z performance on the following intersections:   
 
\item \cite{budavari2009unified} (Unified photo-z paper)
\end{itemize}


\section{Model}
This section describes the probabilistic model for spectroscopic and photometric observations.  

\subsection{Stochastic Process Model of Spectra}
The SED of a quasar is a nonnegative valued function of wavelength.  We model it in rest-frame as a linear combination of a set of basis functions.  We place a log-Gaussian process prior on each of these basis functions.  The generative process for quasar spectra is 

\begin{align}
  \beta_k(\cdot) &\sim \mathcal{GP}(0, K_\theta) \\
  B_k(\cdot) &= \frac{\exp(\beta_k(\cdot))}{\int_\mathcal{W} \exp(\beta_k)}   \\
  \mathbf{w} &\sim p(\mathbf{w}) \, , \text{ s.t. } \sum_{w_k} = 1  \\
  f^{(rest)}_n &= \sum_{k} w_k B_k &&\text{ (SED) }\\
  \tilde f^{(rest)}_n &= \ell_n \sum_{k} w_k B_k
\end{align}

\red{Warp input for varying lengthscale}.  


We view the spectrum of a quasar as a (scaled) probability distribution itself, and thus model it as a random measure.  Specifically, we specify a model for the \emph{rest frame} spectrum of a quasar, and explain observations by warping the input measurements by the appropriate red-shift values. 

As the spectrum of a source characterizes much of the information of interest, we take special care creating a structured prior over $f_n(\cdot)$ using available observations of rich spectra. Furthermore, due to the fact that the red-shift value effectively scales the input of this function, we cannot rely on a fixed grid of $\lambda_1, \dots, \lambda_P$ values, but must define a stochastic process to coherently define a probabilistic model.  The rest of this section outlines a model for these spectra and subsequent photometric observations.  


\section{Experiments}


\section{Discussion}



% Acknowledgements should only appear in the accepted version. 
%\section*{Acknowledgments} 
 
%\textbf{Do not} include acknowledgements in the initial version of
%the paper submitted for blind review.

%If a paper is accepted, the final camera-ready version can (and
%probably should) include acknowledgements. In this case, please
%place such acknowledgements in an unnumbered section at the
%end of the paper. Typically, this will include thanks to reviewers
%who gave useful comments, to colleagues who contributed to the ideas, 
%and to funding agencies and corporate sponsors that provided financial 
%support.  


% In the unusual situation where you want a paper to appear in the
% references without citing it in the main text, use \nocite
%\nocite{langley00}

\bibliography{../refs}
\bibliographystyle{icml2015}

\end{document} 


% This document was modified from the file originally made available by
% Pat Langley and Andrea Danyluk for ICML-2K. This version was
% created by Lise Getoor and Tobias Scheffer, it was slightly modified  
% from the 2010 version by Thorsten Joachims & Johannes Fuernkranz, 
% slightly modified from the 2009 version by Kiri Wagstaff and 
% Sam Roweis's 2008 version, which is slightly modified from 
% Prasad Tadepalli's 2007 version which is a lightly 
% changed version of the previous year's version by Andrew Moore, 
% which was in turn edited from those of Kristian Kersting and 
% Codrina Lauth. Alex Smola contributed to the algorithmic style files.  
