\documentclass[aos,preprint]{imsart}
\pdfoutput=1
\usepackage[OT1]{fontenc}
\usepackage[numbers]{natbib}
\usepackage[colorlinks,citecolor=blue,urlcolor=blue]{hyperref}
\usepackage{hypernat}
\usepackage{geometry}
\geometry{
  body={6in, 9in},
  left=1.25in,
  top=1in
}
\setattribute{journal}{name}{}
\setlength{\parskip}{.5\baselineskip}%
\setlength{\parindent}{0cm}

%% shared defs
%\RequirePackage{natbib}
\usepackage{graphicx} % more modern
%\usepackage{epsfig} % less modern
\usepackage{subfigure} 
\usepackage{amsmath,amsfonts,amssymb,bbm}
\usepackage{tikz}
\usetikzlibrary{fit,positioning}
\usepackage[disable]{todonotes}

% For citations
\usepackage{natbib}

% For algorithms
\usepackage{algorithm}
\usepackage{algorithmic}

% As of 2011, we use the hyperref package to produce hyperlinks in the
% resulting PDF.  If this breaks your system, please commend out the
% following usepackage line and replace \usepackage{icml2015} with
% \usepackage[nohyperref]{icml2015} above.
\usepackage{hyperref}
\usepackage{tabularx}

% Packages hyperref and algorithmic misbehave sometimes.  We can fix
% this with the following command.
%\newcommand{\theHalgorithm}{\arabic{algorithm}}
\DeclareMathOperator{\Tr}{Tr}
\newcommand{\R}{\mathbbm{R}}
\newcommand{\mba}{\mathbf{a}}
\newcommand{\mbb}{\mathbf{b}}
\newcommand{\mbx}{\mathbf{x}}
\newcommand{\mbxt}{\tilde{\mathbf{x}}}
\newcommand{\Sigmat}{\tilde{\Sigma}}
\newcommand{\mbz}{\mathbf{z}}
\newcommand{\mbw}{\mathbf{w}}
\newcommand{\mcN}{\mathcal{N}}
\newcommand{\mcP}{\mathcal{P}}
\newcommand{\mcX}{\mathcal{X}}
\newcommand{\eps}{\epsilon}
\newcommand{\trans}{\intercal}
\newcommand{\Ut}{\tilde{U}}
\DeclareMathOperator*{\argmax}{arg\,max}
\newcommand{\angstrom}{\textup{\AA}}
\newcommand{\red}[1]{\textcolor{red}{[TODO: #1]}}
\newcommand{\Nspec}{{N_{\text{spec}}}}
\newcommand{\Nphoto}{{N_{\text{photo}}}}

%%
%% figwidth, half figwidth
%%
\newlength{\figwidth}
\setlength{\figwidth}{6.75in}
\newlength{\halfwidth}
\setlength{\halfwidth}{3.37in}

\setlength{\figwidth}{6in}
\setlength{\halfwidth}{3in}

%% define some URLS
\urldef\acmweb\url{http://people.seas.harvard.edu/~acm/}
\urldef\rpaweb\url{http://people.seas.harvard.edu/~rpa/}
\urldef\kgweb\url{http://kirkgoldsberry.com/}

%
% end preamble/begin doc
%
\begin{document}

\begin{frontmatter}

  \title{A Stochastic Process Model of Quasar Spectral Energy Distributions}
  \runtitle{Model of Quasar Spectral Energy Distributions}

  \begin{aug}
    \author{\fnms{Andrew} \snm{Miller}%
      \ead[label=e1]{acm@seas.harvard.edu}%
      \ead[label=u1,url]{http://people.seas.harvard.edu/\textasciitilde{}acm/}%
      \thanksref{t1, t5}%
      },
    \author{\fnms{Albert} \snm{Wu}%
      \ead[label=e2]{awu@college.harvard.edu}%
      \thanksref{t1}%
      },
    \author{\fnms{Jeffrey} \snm{Regier}%
      \ead[label=e4]{jeff@stat.berkeley.edu}%
      \thanksref{t2}%
      },
    \author{\fnms{Jon} \snm{McAuliffe}%
      \ead[label=e4]{jon@stat.berkeley.edu}%
      \thanksref{t2}%
    },
    %\and
    \author{\fnms{Dustin} \snm{Lang}%
      \ead[label=e4]{dstn@cmu.edu}%
      \thanksref{t3}%
    },
    %\and
    \author{\fnms{Prabhat} \snm{}%
      \ead[label=e4]{prabhat@lbl.gov}%
      \thanksref{t4}%
    },
    \author{\fnms{David} \snm{Schlegel}%
      \ead[label=e4]{djschlegel@lbl.gov}%
      \thanksref{t4}%
    },
    \author{\fnms{Ryan} \snm{Adams}%
      \ead[label=e3]{rpa@seas.harvard.edu}%
      \ead[label=u3,url]{http://people.seas.harvard.edu/\textasciitilde{}rpa/}%
      \thanksref{t1}%
    }

    %front page footnote
    \thankstext{t1}{School of Engineering and Applied Sciences, Harvard University}
    \thankstext{t2}{Department of Statistics, University of California, Berkeley}
    \thankstext{t3}{McWilliams Center for Cosmology, Carnegie Mellon University}
    \thankstext{t4}{Lawrence Berkeley National Laboratory}
    \thankstext{t5}{\acmweb}
    
    %\thankstext{t1}{ \acmweb }
    %\thankstext{t2}{ \lbweb }
    %\thankstext{t3}{ \rpaweb }
    %\thankstext{t4}{ \kgweb }

    %last page author info
    \address{Andrew Miller\\
      School of Engineering and Applied Sciences\\
      Harvard University\\
      Cambridge, MA 02138, USA\\
      \printead{e1}\\
      \printead{u1}}
    
    \address{Ryan Adams\\
      School of Engineering and Applied Sciences\\
      Harvard University\\
      Cambridge, MA 02138, USA\\
      \printead{e3}\\
      \printead{u3}}
    
    \runauthor{A.~Miller et al.}
    
  \end{aug}

\begin{abstract} 
We describe a method for combining two disparate sources of astronomical data, spectroscopy and photometry, which carry information about sources of light (e.g., stars, galaxies, and quasars) at extremely different spectral resolutions. 
Our model treats the spectral energy distribution (SED) of the radiation from a source as a latent variable, hierarchically generating both photometric and spectroscopic observations.  
Furthermore, we view the problem of SED inference as a density estimation problem, placing a flexible nonparametric prior over the SED of a light source that admits a physically interpretable decomposition and allows us to tractably perform Bayesian inference using Markov chain Monte Carlo (MCMC).  
We use our model to predict the distribution of the redshift of a quasar from five-band (low spectral resolution) photometric data, the so called ``photo-z'' problem. 
Our method shows that tools from machine learning and Bayesian statistics allow us to leverage multiple spectral resolutions of information to make accurate predictions with well-characterized uncertainties. 
\end{abstract} 

\begin{keyword}
\kwd{astronomy}
\kwd{Gaussian process}
\kwd{Bayesian statistics}
\kwd{hierarchical model}
\end{keyword}

\end{frontmatter}

%%%%%
%%% Main paper content
%%%%%
\input{content.tex}

\bibliographystyle{plainnat}
\bibliography{../refs}



\end{document} 




